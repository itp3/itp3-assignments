\documentclass{assignment}

\newcommand{\lecture}{Theoretische Physik IX: Superstringtheorie}
\newcommand{\semester}{SS 2015}
\newcommand{\lecturer}{Prof. Dr. Farnsworth}
\newcommand{\sheet}{Solution 87}
\newcommand{\releaseDate}{22. April 2042}

\begin{document}

The Yukawa potential is given by the expression
\begin{align}
    \phi_Y(\vecr) = \frac{\ef{-mr}}{r}
\end{align}
where $r=|\vecr|$ and $m > 0$ is the mass of the particle that mediates
the potential. The inverse mass is a length scale (Compton length)
that determines the range of the potential. If photons had a rest
mass, the Coulomb potential would have to be replaced by the Yukawa
potential. We can see that the Coulomb potential is the limiting case of
$\phi_Y(\vecr)$ in the zero-mass limit (infinite-range limit).

\begin{exercises}

\exercise{Proof of important statement (Written) [2pt]}
In this exercise, we are going to prove the important statement
\begin{align}
    \integralb{0}{1}{s} s^2 + \frac{2}{3} = 1.  \label{statement}
\end{align}
\begin{tasks}
    \task Solve the integral by on the left hand side (you may use
          Mathematica).
    \task Perform the addition to show that equation~\eqref{statement}
          holds.
\end{tasks}

\begin{solution}
    \begin{tasks}
        \task Using Mathematica, we find that $\integralb{0}{1}{s} s^2 = \frac{1}{3}$.
        \task Then, we have $\frac{1}{3} + \frac{2}{3} = \frac{1+2}{3} = 1$. This is the
              result on the right hand side. $\square$
    \end{tasks}
\end{solution}

\newpage

\exercise{Proof of other important statement (Oral)}
In this exercise, we are going to prove the famous ``other important
statement'':
\begin{align}
    \frac{2}{3} + \integralb{0}{1}{s} s^2 = 1
    \label{otherStatement}
\end{align}
\begin{tasks}
    \task Use the techniques which you have learned in the first
          exercise to prove the ``other important statement''.
    \task Use the trick with the commutative law to prove
          equation~\eqref{otherStatement}.
\end{tasks}

\begin{solution}
    \begin{tasks}
        \task In exercise 1a, we have seen that $\integralb{0}{1}{s} s^2 = \frac{1}{3}$.
        \task Then, we have $\frac{2}{3} + \frac{1}{3}$ on the left hand side. Using the
              commutative law, this is equal to $\frac{1}{3} + \frac{2}{3}$. This
              is exactly what we have calculated in 1b! $\square$
    \end{tasks}
\end{solution}

\end{exercises}

\end{document}
